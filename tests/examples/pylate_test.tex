\documentclass[a4paper,spanish,10pt]{article}%
\usepackage[T1]{fontenc}%
\usepackage[utf8]{inputenc}%
\usepackage{lmodern}%
\usepackage{textcomp}%
\usepackage{lastpage}%
\usepackage{geometry}%
\geometry{left=20mm,right=20mm}%
%
\usepackage{mathtools}%
\title{Viga rectangular a flexión {-} ACI 318{-}19}%
\author{StructureLab}%
\date{\today}%
%
\begin{document}%
\normalsize%
\maketitle%
Funcionamiento%

        \begin{equation}
        M_n = A_s \cdot f_y \cdot \left( d - \frac{a}{2} \right)
        \end{equation}
        Donde:
        \begin{itemize}
            \item $A_s$ es el área de la armadura de refuerzo (en \text{mm}$^2$).
            \item $f_y$ es el esfuerzo de fluencia del acero de refuerzo (en MPa).
            \item $d$ es la distancia desde la fibra extrema en compresión hasta el centroide del acero de refuerzo (en mm).
            \item $a$ es la profundidad del bloque de compresión equivalente (en mm), que se determina como:
            \begin{equation}
            a = \frac{\beta_1 \cdot c}{2}
            \end{equation}
        \end{itemize}

        La profundidad $c$ se relaciona con la deformación en el acero de refuerzo, y $\beta_1$ es un factor que depende de la resistencia especificada del concreto, $f'_c$, según las disposiciones del ACI 318-19.
        \begin{equation}
        M_u \leq \phi \cdot M_n
        \end{equation}
        Donde $M_u$ es el momento flector último requerido debido a las cargas aplicadas, y $\phi$ es el factor de reducción de resistencia, cuyo valor depende del tipo de fallo esperado y puede variar entre 0.65 y 0.90.
        %
\end{document}